%%%%%%%%%%%%%%%%%%%%%%%%%%%%%%%%

%% combined packages and macros preamble.

%%%%%%%%%%%%%%%%%%%%%%%%%%%%%%%%

%%%%%%%%%%%%%%%%%%%%%%%%%%%%%%%%
%% %% some of my preamble 'best practices':

%% package xparse allows for 'macro overloading' in a sense but i havent used
%% it yet.

%% natbib, hyperref   used right now ; have caused errors in past

%%%%%%%%%%%%%%%%%%%%%%%%%%%%%%%%



%\usepackage{enumitem}
% \usepackage{newclude}
%\usepackage{versions}
% \usepackage{graphicx}
% \usepackage{times}
%\usepackage{fancyvrb}
% \usepackage{geometry}  %% currently make geometry modifications in each
% specific doc


\usepackage[english]{babel}
\usepackage{amsfonts}
\usepackage{amstext}
\usepackage{amsmath}
\usepackage{amssymb,dsfont,mathtools}
\usepackage{bbm} %% mathbb vs mathbbm?  I use bbm for indicator fns.
\usepackage{amsthm} %% do i want/ use this correctly?
\usepackage{epsfig, subfigure}
%\usepackage{natbib}
%\usepackage{color}
%\RequirePackage[colorlinks,linkcolor=red,citecolor=blue,urlcolor=blue]{hyperref}
% \RequirePackage{hypernat}

% \usepackage{url}

%% table packages
% \usepackage{multirow}
% \usepackage{booktabs}
% \usepackage{slashbox}

% \usepackage[demo]{graphicx}
% \usepackage{caption}
% \usepackage{subcaption}

%\setlength{\textwidth}{5.5in}
%\setlength{\textheight}{21cm}

% \pdfoutput=1
% \newcommand{\com}[1]{{\noindent  \color{blue}#1}}       % blue, magenta
% \newcommand{\comC}[1]{{\noindent  \color{magenta}#1}}
\newcommand{\KR}[1]{{\noindent  \color{red}#1}}
\newcommand{\FB}[1]{{\noindent  \color{blue}#1}}

\newcommand{\comf}[1]{{\noindent \color{blue}#1}}
\newcommand{\remf}[1]{{\noindent \color{magenta}#1}}


%% here here

\let\proglang=\textsf
\newcommand{\pkg}[1]{{\fontseries{b}\selectfont #1}}
\let\code=\texttt


\usepackage{xparse} %% multi-optional-arguments in macros.
\usepackage{etoolbox} %% gives 'ifstrempty' command


% \newtheorem{theorem}{Theorem}[section]
% \newtheorem{proposition}{Proposition}[section]
% \newtheorem{corollary}{Corollary}[section]
% \newtheorem{lemma}{Lemma}[section]
% \newtheorem*{theorem*}{Theorem}
% \newtheorem*{proposition*}{Proposition}
% \newtheorem*{corollary*}{Corollary}
% \newtheorem*{lemma*}{Lemma}

% \theoremstyle{definition}
% \newtheorem{definition}{Definition}[section]
% \newtheorem{remark}{Remark}[section]
% \newtheorem{example}{Example}[section]
% \newtheorem*{definition*}{Definition}
% \newtheorem*{remark*}{Remark}
% \newtheorem*{example*}{Example}
% % \newtheorem{assumption}{Assumption}[section] %% constr approx style deals
% % % poorly with seciton numbering it seems
% %% \newtheorem{assumption}{Assumption}
% % \newtheorem{conjecture}{Conjecture}[section]

% \newtheorem*{assumption*}{\assumptionnumber}
% \providecommand{\assumptionnumber}{}
% \makeatletter % changes the catcode of @ to 11 from 12
% \newenvironment{assumption}[1]
%  {%
%   \renewcommand{\assumptionnumber}{Assumption #1}%
%   \begin{assumption*}%
%   \protected@edef\@currentlabel{#1}%
%  }
%  {%
%   \end{assumption*}
%  }
% \makeatother % changes the catcode of @ back to 12


%\newcommand{\prf}{\begin{proof}{Proof}}







% --------------------------------------------------------------%
% COMMENTS COLORS

%\usepackage[usenames,dvipsnames]{xcolor}

% \newcommand{\remC}[1]{{\noindent \color{Green}#1}}


% \newcommand{\comC}[1]{{\noindent  \color{Maroon}#1}}       % blue, magenta
% % \newcommand{\comJ}[1]{{\noindent   \color{green}#1}}       % blue, magenta
% \newcommand{\comF}[1]{{\noindent   \color{Dandelion}#1}}       % blue, magenta
% \newcommand{\comOld}[1]{{\noindent  \color{OliveGreen}#1}}       % blue, magenta
% % potential colors ( from http://en.wikibooks.org/wiki/LaTeX/Colors):
% Bittersweet, OliveGreen,  BurntOrange, Maroon
% --------------------------------------------------------------%

\usepackage{float}








%%%%%%%%%%%%%%%%%%%%%%%%%%%%%%%%%%%%%%%%%%%%%%%%
%%%%%%%%%%%%%%%%%%%%%%%%%%%%%%%%%%%%%%%%%%%%%%%%
%%%%%%%%%%%%%%%%%%%%%%%%%%%%%%%%%%%%%%%%%%%%%%%%




%% NOTE: xparse provides method to differentiate between no argument and
%% empty argument (IfNoValueTF; use o instead of O{}).  But i'm not using
%% that, don't want it.  In this definition, the arguemnts are sequential, so
%% dont need to test all combinations just for the last one.
% \NewDocumentCommand\testtt{O{}O{}O{}}{
%   \ifstrempty{#1}{
%     N_{[\, ]}
%   }{
%     \ifstrempty{#2}
%     {MISTAKE}
%     {\ifstrempty{#3}{MISTAKE}
%     {N_{[\, ]} \left( {#1}, {#2}, {#3} \right)}}
%   }
% }

% \NewDocumentCommand\Nb{O{}}{
%   \ifstrempty{#1}{
%     N_{[\, ]}
%   }{
%     N_{[\, ]}\left( {#1} \right)
%   }
% }

% \NewDocumentCommand\Nc{O{}}{
%   \ifstrempty{#1}{
%     N
%   }{
%     N\left( {#1} \right)
%   }
% }

% %% just have one argument; typing the commas is easy
% \NewDocumentCommand\cC{O{}}{
%   \ifstrempty{#1}{
%     \mc C
%   }{
%     \mc C \left( {#1} \right)
%   }
% }


\DeclareMathOperator{\vol}{Vol}
\DeclareMathOperator{\Vol}{Vol}
\DeclareMathOperator{\diam}{diam}
\global\long\def\inv#1{\frac{1}{#1}}
\DeclareMathOperator*{\Bias}{Bias}
\DeclareMathOperator*{\MSE}{MSE}
\DeclareMathOperator*{\Var}{Var}
\DeclareMathOperator*{\tr}{tr}
\DeclareMathOperator*{\Cov}{Cov}
\DeclareMathOperator*{\Cor}{Cor}
\DeclareMathOperator*{\sd}{sd}
% \DeclareMathOperator*{\min}{min}
% \DeclareMathOperator*{\max}{max}
%% \mathrm is owrose than declaremathoperator
%\newcommand{\dom}{\mathrm{dom}}
%\newcommand{\sub}{\mathrm{sub}}
\DeclareMathOperator{\dom}{dom}
\DeclareMathOperator{\sub}{sub}
\DeclareMathOperator{\argmin}{argmin}
\DeclareMathOperator{\argmax}{argmax}






















%%%%%%%%%%%%%%%%%%%%%%%%%%%%%%%%%%%%
%%%%%%%%%%%%%%%%%%%%%%%%%%%%%%%%%%%%
%%%%%%%%%  Macros below here
%%%%%%%%%%%%%%%%%%%%%%%%%%%%%%%%%%%%
%%%%%%%%%%%%%%%%%%%%%%%%%%%%%%%%%%%%






\newcommand{\lp}{\left(} %% )
  \newcommand{\rp}{\right)}
\newcommand{\lb}{\left\{} %% )
  \newcommand{\rb}{\right\}}
\newcommand{\ls}{\left[} %% )
  \newcommand{\rs}{\right]}
\newcommand{\la}{\left\langle}
\newcommand{\ra}{\right\rangle}
\newcommand{\lv}{\left\vert}
\newcommand{\rv}{\right\vert}
\newcommand{\lV}{\left\Vert}
\newcommand{\rV}{\right\Vert}
%\newcommand{\|}{\Vert}

\newcommand{\vp}{\varphi}
\newcommand{\grad}{\nabla}
\newcommand{\hess}{\nabla^2}


\newcommand{\ve}[1]{\boldsymbol{#1}}
\newcommand{\mat}[1]{\ve{#1}}

%% \newcommand{\cC}{\mathcal{C}} %% see below
\newcommand{\PP}{{\mathbb P}}
\newcommand{\PPn}{{\mathbb P}_n}
\newcommand{\FFn}{{\mathbb F}_n}
\newcommand{\QQ}{{\mathbb Q}}
\newcommand{\LL}{{\mathbb L}}
\newcommand{\GG}{{\mathbb G}}
\newcommand{\GGn}{{\mathbb G}_n}
\newcommand{\RR}{\mathbb{R}}
\newcommand{\EE}{\mathbb{E}}
\newcommand{\UU}{\mathbb{U}}
\newcommand{\FF}{{\mathbb F}}
\newcommand{\YY}{{\mathbb Y}}
\newcommand{\WW}{{\mathbb W}}
\newcommand{\HH}{{\mathbb H}}
\newcommand{\NN}{\mathbb{N}}
\newcommand{\ZZ}{{\mathbb Z}}



\newcommand{\ib}{\boldsymbol{i}}
\newcommand{\jb}{\boldsymbol{j}}
\newcommand{\Gb}{\boldsymbol{\Gamma}}



\newcommand{\eps}{\varepsilon}
\newcommand{\ud}{\mathrm{d}}
\newcommand{\pp}{\ensuremath{\mathfrak{P}}}
% \newcommand{\ff}{\ensuremath{\mathfrak{F}}}
\newcommand{\fft}{\ensuremath{\tilde{\mathfrak{F}}}}
\newcommand{\ffn}{\ensuremath{{\mathfrak{F}}^{\otimes n}}}
\newcommand{\ffb}{\ensuremath{\ff_{\text{BDD}}}}
\newcommand{\ffs}{\ensuremath{\ff_{\text{SMU}}}}


\newcommand{\one}{\mathbbm{1}}
\newcommand{\bb}[1]{\mathbb {#1}}
\newcommand{\kl}[2]{\rho_{\mathrm{KL}}\left({#1} \, \Vert \, {#2} \right)}
\newcommand{\mc}[1]{\mathcal {#1}}
\newcommand{\bs}[1]{\boldsymbol {#1}}

\newcommand{\cI}{\mathcal{I}}
\newcommand{\cK}{\mathcal{K}}
\newcommand{\cH}{\mathcal{H}}
\newcommand{\bH}{\boldsymbol{H}}
\newcommand{\bX}{\boldsymbol{X}}
\newcommand{\bx}{\boldsymbol{x}}
\newcommand{\bz}{\boldsymbol{z}}



% \newcommand{\argmin}{\mathrm{argmin}}
%%\newcommand{\inf}{\mathrm{inf}}
%%\newcommand{\sup}{\mathrm{sup}}
\newcommand{\limsupp}{{\overline{\mathrm{lim}}}}
\newcommand{\ess}{{{\mathrm{ess}}}}
%% \newcommand{\width}{{\mathrm{width}}}
\newcommand{\SPAN}{{\mathrm{span}}}

\newcommand{\adj}{\operatorname{adj}}
\newcommand{\linaff}{\operatorname{Lin}}
%% \newcommand{\lin}{{\mathrm{lin}}}
\newcommand{\Lin}{\operatorname{Lin}}


%% polar coords; just regular font for now?
\NewDocumentCommand\PC{O{}}{
        \ifstrempty{#1}{
        P^{d-1}
        }{
        P^{#1}
        }
}
%% polar coords; just regular font for now?
\NewDocumentCommand\PCc{O{}}{
        \ifstrempty{#1}{
        \overline{P}^{d-1}
        }{
        \overline{P}^{#1}
        }
}

\NewDocumentCommand\ffnH{O{}}{
  \ifstrempty{#1}{
  \widehat{f}_{n,\bH}
  }{
  \widehat{f}_{#1}
  }
}

\NewDocumentCommand\fftaun{O{}}{
  \ifstrempty{#1}{
  \widehat{f}_{\tau, n}
  }{
  \widehat{f}_{\tau, #1}
  }
}
\NewDocumentCommand\fftau{O{}}{
  \ifstrempty{#1}{
       f_{\tau,0}
  }{
        f_{\tau,#1}
  }
}

\NewDocumentCommand\LLtau{O{}}{
  \ifstrempty{#1}{
  L
  }{
   L_{#1}
  }
}
\NewDocumentCommand\deriv{m O{}}{
  \ifstrempty{#2}{
  \frac{d}{d #1}
  }{
  \frac{d #2}{d #1}
  }
}

\NewDocumentCommand\pderiv{m O{}}{
  \ifstrempty{#2}{
  \frac{\partial}{\partial #1}
  }{
  \frac{\partial #2}{\partial #1}
  }
}
\NewDocumentCommand\derivtwo{m m O{}}{
  \ifstrempty{#3}{
  \frac{\partial^2}{\partial #1 \partial #2}
  }{
  \frac{\partial^2 #3}{\partial #1 \partial #2}
  }
}



\DeclareMathOperator{\LS}{LS}
\DeclareMathOperator{\HDR}{HDR}

\newcommand{\bracketing}{N_{[\,]}} %% deprecated
\newcommand{\iid}{\stackrel{\text{iid}}{\sim}}
\newcommand{\ind}{\stackrel{\text{ind}}{\sim}}





% \newenvironment{solution}{%
%   \par \vspace{.2cm} { \bf Solution:}%
% }{%
%   \vspace{.2cm}
% }

% \newenvironment{longform}{%
%   \par \vspace{.2cm} { \bf (Begin) Notes to self:}%
% }{%
%         \bf (End) Notes to self:  \vspace{.2cm}
% }
\newenvironment{longform}{%
        \color{gray}
}{\ignorespacesafterend}


%%% Local Variables:
%%% mode: plain-tex
%%% TeX-master: "p"
%%% End:
